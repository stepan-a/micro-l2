\documentclass[10pt,a4paper,notitlepage,twocolumn]{article}
\usepackage{amsmath}
\usepackage{amssymb}
\usepackage{amsbsy}
\usepackage{bbm}
\usepackage{float}
\usepackage[french]{babel}
\usepackage{graphicx}
\usepackage{enumerate}
\usepackage[active]{srcltx}
\usepackage{scrtime}

\newcounter{xnumber}
\setcounter{xnumber}{0}

\newcommand{\exercice}{\textsc{\textbf{Exercice}}
\textbf{\addtocounter{xnumber}{1}\thexnumber}\,\,}
\newcommand{\question}[1]{\textbf{(#1)}}
\setlength{\parindent}{0cm}

\begin{document}

\title{\textsc{Microéconomie\\ \small{(L2, Le Mans)}}}
\author{Stéphane Adjemian\thanks{Université du Mans. \texttt{stephane
DOT adjemian AT univ DASH lemans DOT fr}}}
\date{Mardi 17 décembre 2024}

\maketitle

\thispagestyle{empty}

\bigskip

\begin{quote}
  \textit{Les réponses non commentées ou insuffisamment détaillées ne
  seront pas considérées. Prenez le temps d'écrire des phrases.}
\end{quote}

\bigskip
\bigskip

\question{1} On suppose que les préférences d'un consommateur
représentatif pour la consommation d'un bien sont caractérisées par~:
\[
  u(q) =
  \begin{cases}
    \left( a - \frac{1}{2}q \right)q &\text{ si } q\leq a\\
    \frac{a^2}{2} &\text{ sinon.}
  \end{cases}
\]
où $q$ est la quantité de bien consommée. Calculer l'utilité
marginale et justifier la forme de la fonction d'utilité.\newline

\question{2} On suppose que le consommateur dispose d'un revenu $R>0$
et que le prix unitaire du bien est $p>0$. Déterminer la fonction de
demande inverse, notée $P(q)$, puis la représenter
graphiquement. Montrer que la fonction de demande est~:
\[
  D(p) = \max\left\{ a-p, 0 \right\}
\]
et la représenter graphiquement.\newline

\question{3} Donner une interprétation au paramètre $a$.\newline

\question{4} Le bien est produit par une entreprise dont la fonction
de coût est linéaire~: $C(q) = c q$, avec $c>0$. Quelle hypothèse
faut-il poser sur les paramètres $a$ et $c$~? Déterminer le prix et la
quantité de bien échangée, notées $p^\star$ et $q^\star$, socialement
optimal. Calculer le surplus des consommateurs, le profit des firmes
et le bien être social.\newline

\question{5} Supposons que la firme soit en situation de monopole sur
le marché du bien. Écrire la condition du premier ordre de la
firme. Expliquer pourquoi il ne serait pas optimal pour la firme de ne
pas égaliser recette marginale et coût marginal.\newline

\question{6} Déterminer la quantité produite par le
monopole, $q^m$, et le prix unitaire du bien $p^m$. Montrer que le
prix du monopole est supérieur au coût marginal et que la quantité
échangée de bien est inférieure à l'optimum social.\newline

\question{7} Calculer le surplus des consommateurs et le profit de la
firme dans cet environnement. Déterminer le bien être social. Comparer
avec l'optimum social. Quel problème pose la présence d'un monopole
sur le marché~?\newline

\question{8} Est-il possible pour la firme de rétablir l'optimum
social sans proposer une tarification au coût marginal~?

\end{document}

%%% Local Variables:
%%% mode: latex
%%% TeX-master: t
%%% End:
