\documentclass[10pt,a4paper,notitlepage,twocolumn]{article}
\usepackage{amsmath}
\usepackage{amssymb}
\usepackage{amsbsy}
\usepackage{bbm}
\usepackage{float}
\usepackage[french]{babel}
\usepackage{graphicx}
\usepackage{enumerate}
\usepackage[active]{srcltx}
\usepackage{scrtime}

\newcounter{xnumber}
\setcounter{xnumber}{0}

\newcommand{\exercice}{\textsc{\textbf{Exercice}}
\textbf{\addtocounter{xnumber}{1}\thexnumber}\,\,}
\newcommand{\question}[1]{\textbf{(#1)}}
\setlength{\parindent}{0cm}

\begin{document}

\title{\textsc{Microéconomie\\ \small{(L2, Le Mans)}}}
\author{Stéphane Adjemian\thanks{Université du Mans. \texttt{stephane
DOT adjemian AT univ DASH lemans DOT fr}}}
\date{Mardi 17 décembre 2024}

\maketitle

\thispagestyle{empty}

\bigskip


\bigskip
\bigskip

\question{1} On suppose que les préférences d'un consommateur
représentatif pour la consommation d'un bien sont caractérisées par~:
\[
  u(q) =
  \begin{cases}
    \left( a - \frac{1}{2}q \right)q &\text{ si } q\leq a\\
    \frac{a^2}{2} &\text{ sinon.}
  \end{cases}
\]
où $q$ est la quantité de bien consommée. L'utilité marginale est~:
\[
  u'(q) =
  \begin{cases}
    a-q &\text{ si } q\leq a\\
    0 &\text{ sinon.}
  \end{cases}
\]
la définition avec une condition sur le niveau de $q$ assure que
l'utilité marginale n'est jamais négative, c'est-à-dire que l'utilité
ne décroît jamais lorsque la quantité consommée augmente.\newline

\question{2} On suppose que le consommateur dispose d'un revenu $R>0$
et que le prix unitaire du bien est $p>0$. Le consommateur maximise son utilité par rapport à l$q$a quantité de bien sous une contrainte budgétaire~:
\[
  \begin{split}
    \max_{\{q\}} M + &\left(a-\frac{q}{2}\right)q \mathbb I_{\{q\leq a\}}(q) + \frac{a^2}{2}\mathbb I_{\{q> a\}}(q)\\
    \underline{s.c}&\,\, M+pq = R
  \end{split}
\]
où $M$ représente l'utilité obtenue par la consommation d'autres biens sur d'autres marchés. En substituant la contrainte dans l'objectif, un programme équivalent est~:
\[
    \max_{\{q\}} R-pq + \left(a-\frac{q}{2}\right)q \mathbb I_{\{q\leq a\}}(q) + \frac{a^2}{2}\mathbb I_{\{q> a\}}(q)
\]
La condition du premier ordre est~:
\[
  p =
  \begin{cases}
    a-q &\text{ si }q\leq a\\
    0 &\text{ sinon.}
  \end{cases}
\]
Il s'agit de la fonction de demande inverse $P(q)$. Pour une représentation graphique dans le plan $(q,p)$ il faut tracer une droite décroissante de pente $-1$ et d'ordonnée à l'origine $a$. On obtient directement la fonction de demande en inversant cette fonction~:
\[
  D(p) = \max\left\{ a-p, 0 \right\}
\]
On obtient facilement une représentation graphique dans le plan $(p,q)$ avec une droite de pente $-1$ et d'ordonnée à l'origine $a$.\newline

\question{3} Le paramètre $a$ est le niveau maximal de l'utilité marginale, c'est donc la disponibilité maximale à payer des consommateurs pour unité de bien.\newline

\question{4} Le bien est produit par une entreprise dont la fonction
de coût est linéaire~: $C(q) = c q$, avec $c>0$. On suppose que $a>c$, autrement le marché n'existe pas. En effet les consommateurs n'accepteront jamais de payer plus que $a$ pour une unité de bien et les entreprises n'accepteront pas de vendre une unité de bien à moins de $c$ (qui représente le coût moyen et le coût marginal). À l'optimum social, le prix doit être égal au coût marginal~: $p^\star = c$. On obtient la quantité optimale en substituant dans la fonction de demande~: $q^\star = a-c$. À l'optimum social le profit des firmes est nul (car la fonction de coût est linéaire)~:
\[
\Pi^\star = (p^\star-c)q^\star = 0
\]
Le surplus des consommateurs est~:
\[
  \begin{split}
    S^\star &= \int_{p^{\star}}^\infty D(p)\mathrm dp\\
            &= \int_{c}^a (a-p)\mathrm dp\\
            &= \frac{(a-c)^2}{2}
  \end{split}
\]
Comme le profit des firmes est nul, on a aussi $W^\star=S^\star=\frac{(a-c)^2}{2}$.\newline

\question{5} Si le marché est servi par une firme en situation de monopole, alors celle-ci maximise son profit en égalisant sa recette marginale et son coût marginal. Le programme du monopole est~:
\[
p^m = \arg\max_{\{p\}} p D(p) - cD(p)
\]
La CPO est~:
\[
\underbrace{D(p^m)+p^m D'(p^m)}_{\text{Recette marginale}} = cD'(p^m)
\]
Si la recette marginale est supérieure (inférieure) au coût marginal,
alors le monopole a intérêt à proposer un prix plus élevé (faible). À
l'optimum on a nécessairement égalisation du coût marginal et de la
recette marginale.\newline

\question{6} À l'optimum du monopole, on a~:
\[
a-p^m - p^m = -c
\]
\[
\Leftrightarrow p^m = \frac{a+c}{2}
\]
Le prix optimal du monopole est la moyenne de $a$ (la disponibilité maximale à payer des consommateurs) et du prix socialement optimal. Comme par hypothèse $a>c$, on a nécessairement $p^m>c$. En substituant dans la fonction de demande, on obtient les quantités échangées~:
\[
q^m = a - \frac{a+c}{2} = \frac{a-c}{2}< q^\star
\]
la quantité échangée de bien est inférieure à l'optimum social.\newline

\question{7} Déterminons le surplus des consommateurs et le profit de la
firme dans cet environnement. On a~:
\[
  \begin{split}
    S^m &= \int_{p^m}^\infty D(p)\mathrm dp\\
            &= \int_{\frac{a+c}{2}}^a (a-p)\mathrm dp\\
            &= \frac{(a-c)^2}{8} < S^\star
  \end{split}
\]
et
\[
\Pi^m = (p^m-c)q^m = \left(\frac{a-c}{2}\right)^2>\Pi^\star
\]
Au total, le bien être social est~:
\[
W^m = S^m+\Pi^m = \frac{(a-c)^2}{8} + \frac{(a-c)^2}{4} = \frac{3(a-c)^2}{8}
\]
Clairement on a $W^m<W^\star$, la présence du monopole est socialement sous optimale~:
\[
W^m - W^\star = -\frac{(a-c)^2}{8}<0 
\]
Dans un environnement monopolistique le profit augmente, mais le
surplus baisse plus, au total le bien être social diminue. C'est en ce
sens que la présence du monopole est problématique~: le monopole ne
mange pas efficacement le surplus des consommateurs.\newline

\question{8} Oui si le monopole est capable de parfaitement
discriminer par les prix en proposant un prix spécifique à chaque
consommateur (ou pour chaque unité de bien vendue). S'il propose un
prix égal au surplus de chaque consommateur, alors il peut absorber la
totalité du surplus des consommateurs sans générer de perte
sociale. Le dernier consommateur rentrant sur le marché est celui pour
lequel la disponibilité à payer est exactement égale au coût
marginal. Ainsi, en adoptant cette tarification, la quantité échangée
sur le marché correspond à l'optimum social.

\end{document}

%%% Local Variables:
%%% mode: latex
%%% TeX-master: t
%%% End:
