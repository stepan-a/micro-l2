\documentclass[10pt,a4paper,notitlepage]{article}
\usepackage{amsmath}
\usepackage{amssymb}
\usepackage{amsbsy}
\usepackage{bbm}
\usepackage{float}
\usepackage[french]{babel}
\usepackage{graphicx}
\usepackage{enumerate}

%\usepackage{palatino}

 \usepackage[active]{srcltx}
\usepackage{scrtime}

\newcounter{xnumber}
\setcounter{xnumber}{0}

\newcommand{\exercice}{\textsc{\textbf{Exercice}} \textbf{\addtocounter{xnumber}{1}\thexnumber}\,\,}
\newcommand{\question}[1]{\textbf{(#1)}}
\setlength{\parindent}{0cm}

\begin{document}

\title{\textsc{Microéconomie\\ \small{(Exercices)}}}
\author{Stéphane Adjemian\thanks{Université du Mans. \texttt{stephane DOT adjemian AT univ DASH lemans DOT fr}}}
\date{Le \today\ à \thistime}

\maketitle

\exercice On suppose que les préférences d'un ménage sont représentées par la fonction d'utilité Cobb-Douglas~:
\[
u(x) = x^{\alpha}
\]
avec $0<\alpha<1$, où $x$ est la quantité de bien consommé. Pour chaque unité du bien demandée, le ménage doit payer $p>0$. Le ménage dispose d'un revenu $R$. \question{1} Quelle est la fonction de demande associée à cette fonction d'utilité~? On notera la fonction de demande $D(p)$. \question{2} Calculer l'élasticité, $\epsilon(p)$, de la demande (par rapport au prix). Discuter la forme de la fonction et l'effet d'une variation du paramètre $\alpha$ en interprétant ce paramètre par rapport à l'utilité marginale. \question{3} Calculer le surplus du consommateur, $\Psi(\bar p)$. Représenter le surplus graphiquement en considérant le cas $\alpha=\frac{1}{2}$.\newline

\bigskip

\exercice On suppose que les préférences d'un ménage sont représentées par la fonction d'utilité quadratique~:
\[
u(x) =
\begin{cases}
  x(a-x) & \text{si }x\leq\frac{a}{2}\\
  \frac{a^2}{4} & \text{sinon.}
\end{cases}
\]
avec $a>0$, où $x$ est la quantité de bien consommé. Pour chaque unité du bien demandée, le ménage doit payer $p>0$. Le ménage dispose d'un revenu $R$. \question{1} Calculer l'utilité marginale et justifier la forme de la fonction d'utilité. Est-il possible d'écrire cette fonction de façon plus synthétique~? \question{2} Déterminer la fonction de demande, $D(p)$, exprimée par le ménage. Proposer une interprétation du paramètre $a$. \question{3} Calculer l'élasticité prix de la demande. Discuter ses propriétés et donner une représentation graphique (avec $a=1$). \question{4} Calculer le surplus du ménage, $\Psi(\bar p)$, puis donner une représentation graphique.\newline

\bigskip

\exercice On suppose que les préférences d'un ménage sont représentées par la fonction d'utilité logarithmique~:
\[
u(x) = \beta\log (1+x)
\]
avec $\beta>0$, où $x$ est la quantité de bien consommé. Pour chaque unité du bien demandée, le ménage doit payer $p>0$. Le ménage dispose d'un revenu $R$. \question{1} Déterminer la demande (non négative) exprimée par le ménage. On notera $D(p)$ la fonction de demande. Donner une interprétation au paramètre $\beta$. \question{2} Calculer l'élasticité prix de la demande, $\epsilon(p)$, discuter ses propriétés et représenter graphiquement l'élasticité (en posant $\beta=1$). \question{3} Calculer le surplus du consommateur, $\Psi(\bar p)$, puis donner une représentation graphique.\newline

\bigskip

\exercice On suppose que les préférences d'un ménage sont représentées par la fonction d'utilité exponentielle~:
\[
u(x) = -e^{-\gamma x}
\]
avec $\gamma>0$, où $x$ est la quantité de bien consommé. Pour chaque unité du bien demandée, le ménage doit payer $p>0$. Le ménage dispose d'un revenu $R$. \question{1} Déterminer la demande (non négative) exprimée par le ménage. On notera $D(p)$ la fonction de demande. Donner une interprétation au paramètre $\gamma$. \question{2} Calculer l'élasticité prix de la demande, $\epsilon(p)$, discuter ses propriétés et représenter graphiquement l'élasticité (en posant $\gamma=1$). \question{3} Calculer le surplus du consommateur, $\Psi(\bar p)$, puis donner une représentation graphique.\newline


\end{document}


%%% Local Variables:
%%% mode: latex
%%% TeX-master: t
%%% End:
