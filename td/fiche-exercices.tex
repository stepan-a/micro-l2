\documentclass[10pt,a4paper,notitlepage]{article}
\usepackage{amsmath}
\usepackage{amssymb}
\usepackage{amsbsy}
\usepackage{bbm}
\usepackage{float}
\usepackage[french]{babel}
\usepackage{graphicx}
\usepackage{enumerate}

%\usepackage{palatino}

 \usepackage[active]{srcltx}
\usepackage{scrtime}

\newcounter{xnumber}
\setcounter{xnumber}{0}

\newcommand{\exercice}{\textsc{\textbf{Exercice}} \textbf{\addtocounter{xnumber}{1}\thexnumber}\,\,}
\newcommand{\question}[1]{\textbf{(#1)}}
\setlength{\parindent}{0cm}

\begin{document}

\title{\textsc{Microéconomie\\ \small{(Exercices)}}}
\author{Stéphane Adjemian\thanks{Université du Mans. \texttt{stephane DOT adjemian AT univ DASH lemans DOT fr}}}
\date{Le \today\ à \thistime}

\maketitle

\exercice On suppose que les préférences d'un ménage sont représentées
par la fonction d'utilité Cobb-Douglas~:
\[
u(x) = x^{\alpha}
\]
avec $0<\alpha<1$, où $x$ est la quantité de bien consommé. Pour
chaque unité du bien demandée, le ménage doit payer $p>0$. Le ménage
dispose d'un revenu $R$. \question{1} Quelle est la fonction de
demande associée à cette fonction d'utilité~? On notera la fonction de
demande $D(p)$. \question{2} Calculer l'élasticité, $\epsilon(p)$, de
la demande (par rapport au prix). Discuter la forme de la fonction et
l'effet d'une variation du paramètre $\alpha$ en interprétant ce
paramètre par rapport à l'utilité marginale. \question{3} Calculer le
surplus du consommateur, $\Psi(\bar p)$. Représenter le surplus
graphiquement en considérant le cas $\alpha=\frac{1}{2}$.\newline

\bigskip

\exercice On suppose que les préférences d'un ménage sont représentées
par la fonction d'utilité quadratique~:
\[
u(x) =
\begin{cases}
  x(a-x) & \text{si }x\leq\frac{a}{2}\\
  \frac{a^2}{4} & \text{sinon.}
\end{cases}
\]
avec $a>0$, où $x$ est la quantité de bien consommé. Pour chaque unité
du bien demandée, le ménage doit payer $p>0$. Le ménage dispose d'un
revenu $R$. \question{1} Calculer l'utilité marginale et justifier la
forme de la fonction d'utilité. Est-il possible d'écrire cette
fonction de façon plus synthétique~? \question{2} Déterminer la
fonction de demande, $D(p)$, exprimée par le ménage. Proposer une
interprétation du paramètre $a$. \question{3} Calculer l'élasticité
prix de la demande. Discuter ses propriétés et donner une
représentation graphique (avec $a=1$). \question{4} Calculer le
surplus du ménage, $\Psi(\bar p)$, puis donner une représentation
graphique.\newline

\bigskip

\exercice On suppose que les préférences d'un ménage sont représentées
par la fonction d'utilité logarithmique~:
\[
u(x) = \beta\log (1+x)
\]
avec $\beta>0$, où $x$ est la quantité de bien consommé. Pour chaque
unité du bien demandée, le ménage doit payer $p>0$. Le ménage dispose
d'un revenu $R$. \question{1} Déterminer la demande (non négative)
exprimée par le ménage. On notera $D(p)$ la fonction de
demande. Donner une interprétation au paramètre $\beta$. \question{2}
Calculer l'élasticité prix de la demande, $\epsilon(p)$, discuter ses
propriétés et représenter graphiquement l'élasticité (en
posant $\beta=1$). \question{3} Calculer le surplus du
consommateur, $\Psi(\bar p)$, puis donner une représentation
graphique.\newline

\bigskip

\exercice On suppose que les préférences d'un ménage sont représentées
par la fonction d'utilité exponentielle~:
\[
u(x) = -e^{-\gamma x}
\]
avec $\gamma>0$, où $x$ est la quantité de bien consommé. Pour chaque
unité du bien demandée, le ménage doit payer $p>0$. Le ménage dispose
d'un revenu $R$. \question{1} Déterminer la demande (non négative)
exprimée par le ménage. On notera $D(p)$ la fonction de
demande. Donner une interprétation au paramètre $\gamma$. \question{2}
Calculer l'élasticité prix de la demande, $\epsilon(p)$, discuter ses
propriétés et représenter graphiquement l'élasticité (en
posant $\gamma=1$). \question{3} Calculer le surplus du
consommateur, $\Psi(\bar p)$, puis donner une représentation
graphique.\newline

\bigskip

\exercice Soit une entreprise dont la technologie est caractérisée par
la fonction de production $q=l^2$, où $l$ est le volume d'heures de
travail et $q$ la quantité de bien produite. On note $w$ le salaire
horaire. \question{1} Donner la fonction de coût de
l'entreprise. Calculer le coût moyen et le coût marginal, comparer ces
deux quantités. \question{2} Cette fonction de coût présente-t-elle
des économies d'échelle~?\newline

\bigskip

\exercice Soit une entreprise dont la technologie est caractérisée par
la fonction de production $q=\sqrt{xl}$, où $x$ est la quantité de
matière première utilisée, $l$ est le volume d'heures de travail
et $q$ la quantité de bien produite. On note $w$ le salaire horaire
et $p$ le prix (par unité) de la matière première. \question{1} Donner
la fonction de coût de l'entreprise. Calculer le coût moyen et le coût
marginal, comparer ces deux quantités. \question{2} Cette fonction de
coût présente-t-elle des économies d'échelle~?\newline

\bigskip

\exercice Soit une entreprise dont la technologie est caractérisée par
la fonction de production $q=\sqrt{l}$, où $l$ est le volume d'heures
de travail et $q$ la quantité de bien produite. On note $w$ le salaire
horaire. Pour pouvoir participer au marché la firme doit au préalable
s'acquitter un montant forfaitaire $F>0$. \question{1}
Donner la fonction de coût de l'entreprise. \question{2} Représenter
graphiquement le coût moyen et le coût marginal, comme des fonctions
de $q$, avec $F=1$ et $w=1$. Que se passe-t-il en $q=1$~?\question{3}
Cette fonction de coût présente-t-elle des économies d'échelle~?
Discuter selon le niveau de la production.\newline

\end{document}


%%% Local Variables:
%%% mode: latex
%%% TeX-master: t
%%% End:
